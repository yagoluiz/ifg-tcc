\chapter{Introdução}
\label{cap:intro}

\section{Contextualização}

A cidade de Luziânia, município do estado de Goiás, possui uma população de aproximadamente 200 mil habitantes, sendo a quinta maior cidade do Estado \cite{luziania}. Por esse motivo, a cidade possui inúmeras informações sobre serviços básicos que são essenciais para a população, entre eles, serviços relacionados a educação, lazer, saúde e segurança. Atualmente, essas informações são organizadas em documentos físicos e planilhas eletrônicas, o que permite falhas e inconsistências nos dados.

É necessário organizar essas informações de forma a se criar uma base de dados confiável e segura do município, sendo possível inserir novos dados e também novos serviços. Para organizar os dados, será criado um banco de dados utilizando um Sistema Gerenciador de Banco de Dados (SGBD), a fim de facilitar a criação e manutenção dos dados novos e existentes.

Para realizar a visualização dos dados, será desenvolvido um Sistema de Informação Geográfico Web (SIG Web), fazendo uso das tecnologias mais recentes para o desenvolvimento Web. O sistema além de possuir interação em \textit{browsers} a partir de \textit{desktops} e \textit{notebooks}, será totalmente acessível em dispositivos móveis como \textit{smartphones} e \textit{tablets}, tornando o acesso rápido e eficiente independentemente da plataforma acessada.

Nesse contexto, este trabalho vai possuir um banco de dados geográfico para organizar os dados relacionados a educação, lazer, saúde e segurança, além de um SIG Web para visualizar e cadastrar dados obtidos.

\section{Problema}

Atualmente, a cidade de Luziânia não possui um sistema que gerencie serviços básicos relacionados a educação, lazer, saúde e segurança. Todas as informações são organizadas apenas em documentos físicos e planilhas eletrônicas, o que dificulta o acesso posterior, permitindo possíveis falhas e inconsistências que podem ser geradas com o tempo. Além disso, não existe um banco de dados para armazenamento e manutenção dos dados, principalmente os que indicam localização geográfica.

\section{Objetivos}

\subsection{Objetivo Geral}

O desenvolvimento de um SIG Web para a cidade de Luziânia, de forma a facilitar a organização, o gerenciamento e a visualização de dados relacionados aos serviços de educação, lazer, saúde e segurança.

\newpage

\subsection{Objetivos Específicos}

Para atingir o objetivo geral, este trabalho tem como objetivos específicos:

\begin{itemize}
\item Obter dados da cidade de Luziânia relacionados à educação, lazer, saúde e segurança;
\item Especificar o modelo de dados geográfico de acordo com as orientações nacionais;
\item Implementar o banco de dados geográfico em SGBD com extensão espacial;
\item Inserir os dados obtidos no banco de dados geográfico;
\item Desenvolver um SIG Web para a cidade de Luziânia para visualização e cadastro dos dados obtidos;
\item Possuir interação em \textit{browsers} a partir de \textit{desktops} e \textit{notebooks};
\item Ser acessível em dispositivos móveis como \textit{smartphones} e \textit{tablets}.
\end{itemize}

\section{Justificativa}

As informações de serviços básicos da cidade de Luziânia relacionados a educação, lazer, saúde e segurança não são gerenciados através de um sistema para organização e cadastro de novas informações.
Para organizar esses dados, um Sistema de Informação Geográfico (SIG) é fundamental por conseguir armazenar tanto dados alfanuméricos como geográficos. Obtendo assim uma organização mais segura e eficiente, evitando falhas cadastrais de novos dados e permitindo que a população tenha acesso as informações de forma georreferenciada de uma maneira rápida e sem restrições.

\section{Estrutura}

Este trabalho está dividido nos capítulos apresentados a seguir:

\begin{itemize}
\item O Capítulo 2 apresenta os conceitos de Banco de Dados Geográfico;
\item O Capítulo 3 apresenta os conceitos de Sistema de Informação Geográfico;
\item O Capítulo 4 apresenta os resultados obtidos no trabalho;
\item Por último, o Capítulo 5 apresenta as conclusões do trabalho.
\end{itemize}